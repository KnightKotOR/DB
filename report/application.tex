\chapter{ПРОЕКТИРОВАНИЕ ПРИЛОЖЕНИЯ}
\label{application}
Схема взаимодействия компонент приложения представлена на рисунке \ref{fig:app_structure}. Архитектура программного решения включает следующие основные подсистемы: внешние базы данных (Outer DBs), внутренняя база данных хранения метаданных (Metadata DB), серверное приложение и клиентское веб-приложение.

\begin{enumerate}
	\item Внешние базы данных (Outer DBs) --- произвольный набор реляционных баз данных MySQL, подлежащих анализу структуры.
	Серверное приложение устанавливает соединение с указанными источниками с использованием стандартного протокола SQL через библиотеку mysql.connector.
	Основная функция взаимодействия --- получение информации о количестве записей в базе данных.
	
	\item Metadata DB --- реляционная база MySQL, предназначенная для хранения метаданных;
	
	Функции:
	\begin{itemize}
		\item сохранение извлечённых метаданных;
		\item хранение псевдонимов;
		\item предоставление данных серверу.
	\end{itemize}
	
	\item Серверное приложение реализовано на Python с использованием фреймворков FastAPI (веб-приложение) и Uvicorn (ASGI-сервер). Функции:
	\begin{itemize}
		\item управление подключениями к внешним MySQL-базам данных;
		\item выгрузка метаданных и запись результатов в Metadata DB;
		\item чтение метаданных из Metadata DB;
		\item логика присвоения псевдонимов внешним базам данных и их таблицам;
		\item обработка HTTP-запросов клиентского приложения и генерация ответов в формате JSON.
	\end{itemize}
	
	\item Клиентское веб-приложение разработано на JavaScript с использованием React. Взаимодействие с сервером организовано через библиотеку axios по протоколу HTTP. Функции:
	\begin{itemize}
		\item отправка запросов на получение метаданных и статистической информации;
		\item визуализация результатов и предоставление интерфейса пользователю;
		\item инициирование операций по добавлению новых баз данных в базу метаданных;
		\item инициирование операций по добавлению псевдонимов в базу метаданных и удалению из нее;
		\item инициирование запросов вида SELECT FROM WHERE с учетом возможности использования псевдонимов.
	\end{itemize}
\end{enumerate}

\begin{figure}
	\includegraphics[width=\linewidth]{func_blocks_scheme}
	\caption{Функциональная схема взаимодействия компонент приложения.}
	\label{fig:app_structure}
\end{figure}

