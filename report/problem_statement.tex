\chapter*{}
\begin{center}
	\textbf{ПОСТАНОВКА ЗАДАЧИ}
\end{center}
\label{problem_statement}
\addcontentsline{toc}{chapter}{ПОСТАНОВКА ЗАДАЧИ}


\par Цель курсовой работы — разработать масштабируемое программное обеспечение для автоматического извлечения, хранения и управления метаданными баз данных с поддержкой пользовательским псевдонимов для обращения к хранимым базам.

\par Задачи:

\begin{enumerate}
	\item Предусмотреть возможность подключения и работы с несколькими базами данных MySQL, содержащими произвольные структуры таблиц и данных.
	\item Разработать и реализовать схему базы данных для хранения метаданных, включающую информацию о:
	\begin{itemize}
		\item базах данных,
		\item таблицах,
		\item сущностях,
		\item ключах.
	\end{itemize}
	\item Реализовать механизм автоматизированной выгрузки метаданных из подключаемых баз данных и сохранения их в разработанную базу.
	\item Реализовать интерфейс, позволяющий пользователю задавать псевдонимы для баз данных и их таблиац, а также выполнять запросы вида SELECT FROM WHERE с использованием данных псведонимов. 
\end{enumerate}