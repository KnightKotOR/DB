\chapter* {}
\begin{center}
	\textbf{ВВЕДЕНИЕ}
\end{center}
\label{intro}
\addcontentsline{toc}{chapter}{ВВЕДЕНИЕ}

\par В эпоху взрывного роста количества и разнообразия информации критически важным становится вопрос грамотного управления данными о структуре баз данных. Метаданные — то есть информация, описывающая другие данные, — выступают ключевым инструментом для обеспечения прозрачности процессов, контроля за ресурсами и их эффективного использования. Для современных информационных систем необходимы инструменты, позволяющие автоматизировать сбор, хранение и анализ таких сведений, что напрямую влияет на качество администрирования данных и обоснованность принимаемых решений.

\par Целью данной курсовой работы является проектирование и программная реализация специализированной базы метаданных. Разрабатываемая система будет предназначена для автоматического извлечения, каталогизации и управления метаданными из различных источников. Итогом работы станет универсальное программное решение, умеющее взаимодействовать с множеством баз данных MySQL, независимо от сложности и особенностей их схем.