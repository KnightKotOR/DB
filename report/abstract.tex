\chapter*{}
\thispagestyle{empty}

\begin{center}
	\textbf{РЕФЕРАТ}
\end{center}

На 17~с., 9~рисунков, 2~приложения.


КЛЮЧЕВЫЕ СЛОВА: Метаданные, базы данных, СУБД, MySQL, клиент-серверная архитектура, веб-приложение.


Тема курсовой работы: "<Разработка и программирование базы метаданных">

Предметом исследования курсовой работы являются процессы извлечения, хранения и анализа метаданных баз данных, а также средства автоматизации управления информацией о структуре реляционных БД.

Целью курсовой работы является создание программного решения, способного автоматически подключаться к произвольным базам данных MySQL, извлекать их  метаданные, сохранять их в специализированное хранилище и предоставлять пользователю средства просмотра статистики.


В рамках курсовой работы спроектирована и реализована база метаданных, охватывающая сведения о базах данных, таблицах, столбцах и ключах; разработан серверный модуль, автоматически извлекающий метаданные из INFORMATION\_SCHEMA и сохраняющий их в созданную структуру; создано хранилище временных срезов статистики на базе InfluxDB; реализован веб-интерфейс на React для просмотра метаданных, получения статистики и сохранения её во базу данных временных рядов.

\begin{center}
	\textbf{ABSTRACT}
\end{center}
17 pages, 9 figures, 2 appendices


KEYWORDS: Metadata, databases, MSDB, MySQL, client-server architecture, web application.

Title of the thesis: "Development and programming of a metadata database."

The subject of this coursework is the processes of extracting, storing, and analyzing database metadata, as well as tools for automating the management of information about the structure of relational databases.

The purpose of the coursework is to create a software solution capable of automatically connecting to arbitrary MySQL databases, extracting their metadata, storing it in a specialized repository, and providing the user with tools for viewing statistics.

As part of the coursework, a metadata database was designed and implemented, covering information about databases, tables, columns, and keys; a server module was developed that automatically extracts metadata from INFORMATION\_SCHEMA and saves it to the created structure; a storage for temporary statistics snapshots based on InfluxDB was created; a web interface based on React was implemented for viewing metadata, obtaining statistics, and saving them to a time series database.

\newpage
\endinput