\documentclass[12pt]{article}
\usepackage[russian]{babel}
\usepackage{indentfirst}
\usepackage [left=30 mm, top=15 mm, right=30 mm, bottom=20mm, nohead, footskip=10 mm] {geometry}

\parindent=24pt

\begin{document}
	\thispagestyle{empty}
	
	\begin{center}
		\large{МИНОБРНАУКИ РОССИИ} \par
		\vspace{0.3cm}
		\normalsize
		{ФЕДЕРАЛЬНОЕ ГОСУДАРСТВЕННОЕ АВТОНОМНОЕ ОБРАЗОВАТЕЛЬНОЕ УЧРЕЖДЕНИЕ ВЫСШЕГО ОБРАЗОВАНИЯ} \par
		\vspace{0.3cm}
		\textbf{\guillemotleft САНКТ-ПЕТЕРБУРГСКИЙ ПОЛИТЕХНИЧЕСКИЙ}
		\textbf{УНИВЕРСИТЕТ ПЕТРА ВЕЛИКОГО \guillemotright} \par
		\vspace{0.3cm}
		
		{Институт компьютерных наук и кибербезопасности}\par
		{Высшая школа технологий искусственного интеллекта}\par
		{Направление 02.03.01 Математика и Компьютерные науки}	
	\end{center}
	
	\vfill
	
	\begin{center}
		{\LARGE КУРСОВАЯ РАБОТА} \par
		\vspace{0.3cm}
		{\large по дисциплине \guillemotleft Управление знаниями и технологии баз данных\guillemotright}\par
		{\LARGE Разработка и программирование базы метаданных\\ }\par
	\end{center}
	
	\vfill
	
	\begin{flushleft}
		Студент: \hspace{1.8cm} \rule[0pt]{2.5cm}{0.5pt}\hfill Сергиенко Кирилл Александрович\par
		\vspace{1.5cm}
		Преподаватель: \hspace{0.55cm} \rule[0pt]{2.5cm}{0.5pt}\hfill  Попов Сергей Геннадьевич
	\end{flushleft}
	
	\vspace{0.5cm}
	
	\begin{flushright}
		\guillemotleft \rule[0pt]{0.8cm}{0.5pt}\guillemotright \rule[0pt]{2cm}{0.5pt} 20\rule[0pt]{0.5cm}{0.5pt} г.
	\end{flushright}
	
	\vfill
	
	\begin{center}
		Санкт-Петербург -- 2025
	\end{center}
	

	
	
	\thispagestyle{empty}
	\newpage
	\tableofcontents
	
	
	
	\newpage
	\section{Цель и задачи}
	
	\par Цель — разработать масштабируемое программное обеспечение для автоматического извлечения, хранения и управления метаданными баз данных с поддержкой пользовательским псевдонимов для обращения к хранимым базам.
	
	\par Задачи:

	\begin{enumerate}
		\item Разработать и реализовать схему базы данных для хранения метаданных, включающую информацию о:
		\begin{itemize}
			\item серверах баз данных,
			\item самих базах данных,
			\item таблицах,
			\item сущностях,
			\item составных ключах.
		\end{itemize}
		\item Реализовать механизм автоматизированной выгрузки метаданных из подключаемых баз данных и сохранения их в разработанную базу.
		\item Реализовать интерфейс, позволяющий пользователю указывать суррогатные имена (псевдонимы) для баз данных и выполнять запросы с использованием данных имен. 
	\end{enumerate}
	
\end{document}